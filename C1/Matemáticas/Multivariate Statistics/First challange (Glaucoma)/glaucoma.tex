\documentclass{article}
\usepackage{graphicx}
\usepackage{wrapfig}
\usepackage{float}
\usepackage{amssymb}
\usepackage[T1]{fontenc}
\usepackage[parfill]{parskip}
\usepackage[utf8]{inputenc}
\usepackage{imakeidx}
\usepackage{amsmath}
\usepackage{listings}
\usepackage{tcolorbox}
\usepackage{makecell}
\usepackage{amsthm}
\usepackage{tikz}  %Para hacer diagramas
\usepackage[rightcaption]{sidecap} %Escribir a la derecha de una imagen
\usepackage{multirow,tabularx}
\tcbuselibrary{listingsutf8}
\usepackage[a4paper]{geometry}
\usepackage{amssymb}
\usepackage{datetime}
\geometry{top=2.54cm, bottom=2.54cm, left=2.54cm, right=2.54cm}
\setlength\parindent{0pt}
\begin{document}
\begin{titlepage}
\centering
\vspace{1cm}
{\includegraphics[width=0.7\textwidth]{logo-ugr.png}\par}
\vspace{1cm}
{\bfseries\LARGE University of Granada \par}
\vspace{1cm}
{\scshape\Huge \textbf{Exploring the data of 121 patients who underwent glaucoma surgery. Conclusions.}  \par}
\vspace{1cm}
{\itshape\Large Multivariate Statistics \par}
\vspace{1.2cm}
{\Large Author:\par}

Quintín Mesa Romero \par

\end{titlepage}


\section{Introduction}

% In this section we introduce the problem we have studied and why it is interesting to solve it. Furthermore,
% we explain the problem's context, what do we know about it, what we don't know, and finally, what it wouldd 
% represent if the results that wh have obtained, throw light to that last question.

Glaucoma is a leading cause of irreversible blindness worldwide. It is a disease of a progressive op-
tic neuropathy with loss of retinal neurons and their axons, which can result in blindness in case of
untreatment[1][2]. In short, it is a group of diseases that kill retinal ganglion cells [2].

The strongest known risk factor is high IOP (Intraocular Pressure), but it is not the only factor responsi-
ble for glaucoma [2]. In fact, people with myopia greater than five dioptres, people aged 60 years or more,
people with thin cornea, and even people with different skin type, such as africans or afro-caribbean are
more likely to develop glaucoma. Of course, having family history multiplies the risk of developing the
desease [3].

Given that there are about 80 million people suffering from glaucoma, and it is estimated that over 112
million individuals will have it by 2040, it is reasonable to think that there is a treatment or an operation
to reverse the disease before it is too late [4]. Indeed, there is a surgery based on laser technology which is 
being applied to people with glaucoma[5]. 

From a research point of view, in relation to this surgery, we wonder whether the pre-surgery condition is related, 
in any way, to the long-term progression. 
For this purpose, we have studied a dataset with information of 121 patients who underwent glaucoma surgery 
using laser technology, in which variables have been meassured before and after the surgery over a three-month 
period at different time intervals.

Studying this relationship is crucial, primarily because it can inform better treatment strategies and patient 
outcomes, and clinicians can better predict which patients are more likely to benefit from laser surgery versus 
those who may need alternative or additional interventions, in order to improve their lives.


\section{Methods and techniques}

\subsection{Data Collection, Preparation and Cleaning}

The data from the 121 patients have been stored in an Excel file. This dataset includes clinical variables
related to presurgical conditions and post-surgical outcomes after the laser-based surgery.
The dataset has been processed, prepared and cleaned using the \textbf{R} statistical environment.

First of all, we have to load the library \textbf{readxl}:

\begin{center}
    {\includegraphics[width=0.7\textwidth]{imgs/img1.png}\par}
\end{center}


Now, we are ready to load the dataset. For this purpose, we use the function \textbf{read\_excel}:

\begin{center}
    {\includegraphics[width=0.7\textwidth]{imgs/img2.png}\par}
\end{center}


Once the dataset is loaded, it is advisable to check the structure of the data, and for this, we
use \textbf{str}:

\begin{center}
    {\includegraphics[width=0.6\textwidth]{imgs/img3.png}\par}
\end{center}

Also, in order to not to alter the original dataset, it is advisable to make a copy and work with it in the
future. We will load the variable names from the dataset too. For that, we make:

\begin{center}
    {\includegraphics[width=0.6\textwidth]{imgs/img5.png}\par}
\end{center}

It is easy to see that there are missinng values, so we have to handle them. We can use the 
function \textbf{colSums} with the argument \textbf{is.na(data\_glaucoma)}, which tells us the number of
NA values that there are for each variable:

\begin{center}
    {\includegraphics[width=0.6\textwidth]{imgs/img4.png}\par}
\end{center}


There are different options for handling \textbf{NA} data. One could be simply removing the rows with missing values, but
it might lead to the loss of valuable data and even statistical power. Other option is to replace missing values
with the mean or the \textbf{median}. We will used the last one; we will replace the missing data by the median of the
corresponding variable. To do this, we have made a for loop in which for each variable with missing values, each of
these NA values are replaced by the median of the rest of the values of the corresponding variable. 

\begin{center}
    {\includegraphics[width=0.85\textwidth]{imgs/img6.png}\par}
\end{center}

We check the change:

\begin{center}
    {\includegraphics[width=1\textwidth]{imgs/img7.png}\par}

    {\includegraphics[width=0.85\textwidth]{imgs/img8.png}\par}
\end{center}


Then, we convert the \textbf{categorical} variables of the dataset into \textbf{factor} type:

\begin{center}
    {\includegraphics[width=0.85\textwidth]{imgs/img9.png}\par}
\end{center}

Once all these changes have been applied to the data, we obtain:

\begin{center}
    {\includegraphics[width=0.85\textwidth]{imgs/img11.png}\par}
    {\includegraphics[width=0.85\textwidth]{imgs/img10.png}\par}
\end{center}

Finally, since there are some quantitative variables with non-numerical values (which makes them to be 
considered by R as char type), I have decided to subsitute these non-numerical values by the mediane of 
the rest of the numeric values of the variable, instead of removing them, which would have been another
solution to this problem.

The quantitative variables that R has considered are:

\begin{center}
    {\includegraphics[width=0.85\textwidth]{imgs/img23.png}\par}
\end{center}

We need to do the substitution to the variables  CUADRANTES, PIO\_NORMAL, PIO\_1\_MES, FARMACOS\_PRE. For that purpose, we do:

\begin{center}
    {\includegraphics[width=0.85\textwidth]{imgs/img24.png}\par}
    {\includegraphics[width=0.85\textwidth]{imgs/img25.png}\par}
    {\includegraphics[width=0.85\textwidth]{imgs/img27.png}\par}
    {\includegraphics[width=0.85\textwidth]{imgs/img26.png}\par}
\end{center}

Eventually, the structure of the dataset is as follows:

\begin{center}
    {\includegraphics[width=0.85\textwidth]{imgs/img28.png}\par}
\end{center}

\subsection{Outliers detection and treatment}

With regard to the detection of outliers, if we are studying quantitative variables, an informative
way to detect them is with a boxplot graphic. However, this is not the case with categorical 
variables.

\begin{center}
    {\includegraphics[width=0.85\textwidth]{imgs/img22.png}\par}
\end{center}

So, we need something that throws light to the outliers of all the variables. The decision we have 
taken about outliers is that we are going to modify them by the mean of their variables. 

For the purpose of detecting the outliers, we will use the IQR (Interquartile Range), which is 
defined as the difference of the third and the firt quartiles ($Q_{3}-Q_{1}$), by means of a function
taken from the practice guide, which detects the outliers in each variables and subsitutes then by
the mean of their variables:
\begin{center}
    {\includegraphics[width=0.85\textwidth]{imgs/img29.png}\par}
\end{center}

We apply that function to all the quantitative variables:

\begin{center}
    {\includegraphics[width=0.85\textwidth]{imgs/img30.png}\par}
\end{center}

and obtain:

\begin{center}
    {\includegraphics[width=0.7\textwidth]{imgs/img31.png}\par}
\end{center}


\subsection{Study of Correlated Variables}

Once all data from the dataset is treated, we can visualize its correlation matrix, in order to draw 
conclusions according to the variables.

\begin{center}
    {\includegraphics[width=1.0\textwidth]{imgs/img32.png}\par}
\end{center}

In order to illustrate the information given by the correlation matrix, Here 
we present the heatmap of the correlation matrix, where the most related variables 
are shown in reddish tones:
\begin{center}
    {\includegraphics[width=0.6\textwidth]{imgs/img33.png}\par}
\end{center}

Visually, we can infer which variables are related the most, but, we prefer to extract that
information easily, so, taking advantage of the fact that R is a programming language, we do what follows:

\begin{center}
    {\includegraphics[width=0.85\textwidth]{imgs/img34.png}\par}
\end{center}

and obtain:

\begin{center}
    {\includegraphics[width=0.85\textwidth]{imgs/img35.png}\par}
\end{center}
\hspace{1cm}

We observe a certain relationship between the variables 
\begin{itemize}
    \item (N\_IMPACTOS,ENERGIA\_TOTAL), (ENERGIA\_IMPACTO,ENERGIA\_TOTAL)
    \item (EDAD, PIO\_NORMAL), (EDAD\_PIO\_1\_MES), (EDAD, PIO\_3\_MES), (EDAD, FARMACOS\_1\_MES)
    \item (PIO\_1\_SEMANA, PIO\_1\_MES), (PIO\_1\_MES, PIO\_3\_MES), (PIO\_1\_MES,FARMACOS\_1\_MES), 
    (FARMACOS\_1\_MES,FARMACOS\_3\_MES), (FARMACOS\_PRE,FARMACOS\_1\_MES)
\end{itemize}

Once we have all this information about the variables, we should interpret it, in order to draw some 
conclusions about the question we made at the beginning.

\hspace{1cm}
\subsection{Data Analysis with R}

It can be interesting to keep an eye on the correlations of the variable \textbf{EDAD} and 
\textbf{PIO} variables, as both are factors of risk when suffering from glaucoma. Also, studying 
a bit more the relation between que \textbf{PIO} variables, could give us more information about
the increasing/decreasing of the Intraocular Pressure after the surgery. 
Also, it would be interesting to interpret the realtion between \textbf{PIO} and \textbf{FARMACOS}.

For this pursope, I have written R code, with the help of some blogs, books, tutorials, whose references
I leave below, and also, I have used ChatGPT AI for making better the structure of the code and to know
the meaning and use of some funtions of R which are a bit advanced compared to my R skills. I know that I
could have done it easily, with the tools I have at the moment, but I preferred to investigate the R language 
and discover all the potential it has, because it will be beneficial for my future work.

\hspace{1cm}

\begin{center}
    {\includegraphics[width=0.85\textwidth]{imgs/img36.png}\par}
\end{center}

\hspace{1cm}

Explanation of the code above:

Since we want to analyze the effect of age on intraocular pressure, we have added a column called
"AGE\_GROUP", whose values are ranges of age, between 0 and more than 71. For that purpose, we have categorized 
the variable AGE by means of the function cut. Once we have the age groups, for each one we calculate the 
mean intraocular pressure, for what we use the functions group\_by and summarise from the library dplyr, which enable us to 
calculate descriptive statistics, such us means, sums, etc., for each group, according to the chosen variables.

Now, we have the intraocular pressure of each age group, which would give us significant information.

Then, we want to identify the number of patients who experienced an increase in intraocular pressure after the
surgery. So, for that motivation, we use the function mutate from the library dplyr, which enables us to add new columns to 
a data frame, or modify existing columns,on the basis of the values of other columns. With this function, we have added two new columns
related to the increase of the PIO, from the first week to the first moth, and from the first month to the third.
After that, we have calculated the number of patients who experienced the increase of PIO, by means of the function filter, from 
dplyr, which, given a condition, shows the data that satisfies it.

After that, we want to dig into the relationship between PIO and medication, for what we have extracted the correlation
values between these variables (despite of the fact that we have just obtained them above, we wanted to use the function cor, applied to 
concrete variables, instead to all the quantitative, as we did before). These correlations will give us pretty significant information.

Finally, as a visual support, we have used a boxplot to see the distribution of PIO across age groups.

The result of the execution of the code is the following:

\begin{center}
    {\includegraphics[width=0.85\textwidth]{imgs/img37.png}\par}
    {\includegraphics[width=0.85\textwidth]{imgs/img38.png}\par}
    {\includegraphics[width=0.85\textwidth]{imgs/img39.png}\par}
\end{center}


\section{Results \& Conclusions}

Once we have analyzed the dataset, the correlation of the variables, and we have extracted significant information,
we are ready to make a series of statements that will lead us to answer the question we asked ourselves in the beginning: 
whether the pre-surgery condition is related, in any way, to the long-term progression.

\begin{itemize}
    \item The trend observed in mean intraocular pressure according to age groups \textbf{reinforces the idea that older 
    patients are more likely to have higher IOP}. This suggests that preoperative conditions should be carefully 
    evaluated in this group, as their age may increase the risk of postoperative complications.
    \item It has been observed that \textbf{36 patients (around 30\%) experienced an increase of the IOP after the surgery}. This 
    suggests that the surgery does not result in the expected improvement in IOP. This suggests that patients with more serious
    pre-existing conditions are more susceptible to experience an increase in postoperative IOP.
    \item The moderate \textbf{correlation between the use of  preoperative medication and the use of them after 1 month} (0.58), 
    shows that patients who required more drugs before the surgery, tend to continue to need medication after the surgery. 
    This suggests that preoperative conditions may be a good predictor of the need for subsequent treatment and, therefore, the 
    efficacy of surgery may be influenced by the patient's preoperative conditions.
    \item The \textbf{correlation between PIO and the use of medication at the firt month} (0.56), suggests that patients who present
    a higer IOP after the surgery, tend to need more medication. So, it is possible that patients with serious preoperative conditions 
    need a more aggressive approach.

\end{itemize}

In view of the results obtained and analyzed, we conclude, that there is evidence that conditions prior to surgery have a significant relationship with conditions after 
surgery.

\newpage

\section{References}

\begin{itemize}
    \item [ [1] ] Glaucoma https://www.ncbi.nlm.nih.gov/pmc/articles/PMC8473801/
    \item [[2]] https://eyes.arizona.edu/sites/default/files/glaucoma.pdf
    \item [[3]] https://www.clinicbarcelona.org/en/assistance/diseases/glaucoma/risk-factors-and-causes
    \item [[4]] https://www.glaucomapatients.org/basic/statistics/, 
    https://glaucoma.org/articles/glaucoma-worldwide-a-growing-concern 
    \item [[5]] https://glaucoma.org/treatment/laser
    \item [[6]] https://es.r4ds.hadley.nz/ 
    \item [[7]] https://adv-r.hadley.nz/functions.html
    \item [[8]] https://diytranscriptomics.com/Reading/files/The%20Art%20of%20R%20Programming.pdf
    \item [[9]] https://rsanchezs.gitbooks.io/rprogramming/content/chapter9/

\end{itemize}

\end{document}
