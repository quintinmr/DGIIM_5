\documentclass{article}
\usepackage{graphicx}
\usepackage{wrapfig}
\usepackage{float}
\usepackage{amssymb}
\usepackage[T1]{fontenc}
\usepackage[parfill]{parskip}
\usepackage[utf8]{inputenc}
\usepackage{imakeidx}
\usepackage{amsmath}
\usepackage{listings}
\usepackage{tcolorbox}
\usepackage{makecell}
\usepackage{amsthm}
\usepackage{tikz}  %Para hacer diagramas
\usepackage[rightcaption]{sidecap} %Escribir a la derecha de una imagen
\usepackage{multirow,tabularx}
\tcbuselibrary{listingsutf8}
\usepackage[a4paper]{geometry}
\usepackage{amssymb}
\usepackage{datetime}
\geometry{top=2.54cm, bottom=2.54cm, left=2.54cm, right=2.54cm}
\setlength\parindent{0pt}
\begin{document}
\begin{titlepage}
\centering
\vspace{1cm}
{\includegraphics[width=0.7\textwidth]{logo-ugr.png}\par}
\vspace{1cm}
{\bfseries\LARGE University of Granada \par}
\vspace{1cm}
{\scshape\Huge \textbf{Exploring the data of 121 patients who underwent glaucoma surgery. Conclusions.}  \par}
\vspace{1cm}
{\itshape\Large Multivariate Statistics \par}
\vspace{1.2cm}
{\Large Author:\par}

Quintín Mesa Romero \par

\end{titlepage}


\section{Introduction}

% In this section we introduce the problem we have studied and why it is interesting to solve it. Furthermore,
% we explain the problem's context, what do we know about it, what we don't know, and finally, what it wouldd 
% represent if the results that wh have obtained, throw light to that last question.

Glaucoma is a leading cause of irreversible blindness worldwide. It is a disease of a progressive op-
tic neuropathy with loss of retinal neurons and their axons, which can result in blindness in case of
untreatment[1][2]. In short, it is a group of diseases that kill retinal ganglion cells [2].

The strongest known risk factor is high IOP (Intraocular Pressure), but it is not the only factor responsi-
ble for glaucoma [2]. In fact, people with myopia greater than five dioptres, people aged 60 years or more,
people with thin cornea, and even people with different skin type, such as africans or afro-caribbean are
more likely to develop glaucoma. Of course, having family history multiplies the risk of developing the
desease [3].

Given that there are about 80 million people suffering from glaucoma, and it is estimated that over 112
million individuals will have it by 2040, it is reasonable to think that there is a treatment or an operation
to reverse the disease before it is too late [4]. Indeed, there is a surgery based on laser technology which is 
being applied to people with glaucoma[5]. 

From a research point of view, in relation to this surgery, we wonder whether the pre-surgery condition is related, 
in any way, to the long-term progression. 
For this purpose, we have studied a dataset with information of 121 patients who underwent glaucoma surgery 
using laser technology, in which variables have been meassured before and after the surgery over a three-month 
period at different time intervals.

Studying this relationship is crucial, primarily because it can inform better treatment strategies and patient 
outcomes, and clinicians can better predict which patients are more likely to benefit from laser surgery versus 
those who may need alternative or additional interventions, in order to improve their lives.


\section{Methods and techniques}

\subsection{Data Collection, Preparation and Cleaning}

The data from the 121 patients have been stored in an Excel file. This dataset includes clinical variables
related to presurgical conditions and post-surgical outcomes after the laser-based surgery.
The dataset has been processed, prepared and cleaned using the \textbf{R} statistical environment.

First of all, we have to load the library \textbf{readxl}:

\begin{center}
    {\includegraphics[width=0.7\textwidth]{imgs/img1.png}\par}
\end{center}


Now, we are ready to load the dataset. For this purpose, we use the function \textbf{read\_excel}:

\begin{center}
    {\includegraphics[width=0.7\textwidth]{imgs/img2.png}\par}
\end{center}


Once the dataset is loaded, it is advisable to check the structure of the data, and for this, we
use \textbf{str}:

\begin{center}
    {\includegraphics[width=0.6\textwidth]{imgs/img3.png}\par}
\end{center}

Also, in order to not to alter the original dataset, it is advisable to make a copy and work with it in the
future. We will load the variable names from the dataset too. For that, we make:

\begin{center}
    {\includegraphics[width=0.6\textwidth]{imgs/img5.png}\par}
\end{center}

It is easy to see that there are missinng values, so we have to handle them. We can use the 
function \textbf{colSums} with the argument \textbf{is.na(data\_glaucoma)}, which tells us the number
NA values that there are for each variable:

\begin{center}
    {\includegraphics[width=0.6\textwidth]{imgs/img4.png}\par}
\end{center}


There are different options for handling \textbf{NA} data. One could be simply removing the rows with missing values, but
it might lead to the loss of valuable data and even statistical power. Other option is to replace missing values
with the mean or the \textbf{median}. We will used the last one; we will replace the missing data by the median of the
corresponding variable. To do this, we have made a for loop in which for each variable with missing values, each of
these NA values are replaced by the median of the rest of the values of the corresponding variable. 

\begin{center}
    {\includegraphics[width=0.85\textwidth]{imgs/img6.png}\par}
\end{center}

We check the change:

\begin{center}
    {\includegraphics[width=1\textwidth]{imgs/img7.png}\par}

    {\includegraphics[width=0.85\textwidth]{imgs/img8.png}\par}
\end{center}


Then, we convert the \textbf{categorical} variables of the dataset into \textbf{factor} type:

\begin{center}
    {\includegraphics[width=0.85\textwidth]{imgs/img9.png}\par}
\end{center}

Once all these changes have been applied to the data, we obtain:

\begin{center}
    {\includegraphics[width=0.85\textwidth]{imgs/img11.png}\par}
    {\includegraphics[width=0.85\textwidth]{imgs/img10.png}\par}
\end{center}

\subsection{Outliers Detection and Treatment}



\section{Results}
\section{Conclusions}

\section{References}

\begin{itemize}
    \item [ [1] ] Glaucoma https://www.ncbi.nlm.nih.gov/pmc/articles/PMC8473801/
    \item [[2]] https://eyes.arizona.edu/sites/default/files/glaucoma.pdf
    \item [[3]] https://www.clinicbarcelona.org/en/assistance/diseases/glaucoma/risk-factors-and-causes
    \item [[4]] https://www.glaucomapatients.org/basic/statistics/, 
    https://glaucoma.org/articles/glaucoma-worldwide-a-growing-concern 
    \item [[5]] https://glaucoma.org/treatment/laser

\end{itemize}

\end{document}
